\documentclass[review]{elsarticle}
\usepackage{lineno,hyperref}
\modulolinenumbers[5]

\usepackage[utf8]{inputenc}
\usepackage{amssymb}
\usepackage{fullpage}
\usepackage{amsmath}
\usepackage{lipsum}% http://ctan.org/pkg/lipsum
\usepackage{setspace}% http://ctan.org/pkg/setspace
\usepackage{multirow}
%opening
\title{Summary of POMDP methods}


\begin{document}




\begin{tabular}{ l|l|c|c|c|c|c|c|c|c| }
\cline{2-9}
				  & name 	    			& state & action & observation & belief	 & online	 & offline	& optimisation\\ \hline
POMDP 			      	  & PBVI\cite{PBVI} 			&  d    &   d    & 	d      & d  	 & \checkmark    &              &  \\ \cline{2-9}
				  & SARSOP\cite{SARSOP}   		&  d 	&   d 	 & 	d      & c       & \checkmark    &		&  \\ \cline{2-9} 
				  & MCVI\cite{MCVI}	   		&  c 	&   d 	 & 	d      & c       &     		 & \checkmark	&  \\ \cline{2-9} 
				  & PF-POMDP\cite{NIPS2008_0628}	&  c 	&   c 	 & 	c      & c       & \checkmark    &		&  \\ \hline 
Tree search		  	  & 					&   	&    	 & 	       &         &     	         &		&  \\ \hline  
Planning		          & 					&   	&    	 & 	       &         &     	         &		&  \\ \hline  
Optimal control			  & 					&   	&    	 & 	       &         &     	         &		&  \\ \hline   
\end{tabular}

\paragraph{POMDP}: characterised by a belief state of dimension $|S|$. The belief state grows in exponentially in size, in a $10 \times 10$
state space the belief space has a dimension of $100!$. If reward function is convex the value function can be represented by a set of piece
wise linear and convex functions. This is basically a set of hyperplanes $\Gamm$, each of which must dominate the rest at some sampled point.


Two problems: curse of dimensionality, for a state space of size $100$ the belief state simplex is of dimension $100$. The second obstacle
is the curse of history. If task requires many actions, the planning problem complexity grows exponentially with the time horizon.
If we need a long planning horizon, The optimal belief tree may have O(AO^T) worst case.


The value function is represented by a set of $\alpha$-vectors $ \in \Gamma$. The value function has been proven to be piece-wise 
linear and convex (PWLC) for finite horizon. For infinite horizon case it can be approximated to any precision using PWLC function.

In an infinite-horizon POMDP th value function can be approximated arbitrarily well by a finite set of $\alpha$-vectors.

The problems considers Tag, Hallway, Rock Sample consider discretised state and actions spaces bellow a 100 values.
(Hallway: 93 states, 5 actions, 17 observations)

\paragraph{PBVI}
Keeps growing a set of belief points, but makes sure that these are reachable beliefs. Assumption is that the approximate value function
would generalise well 

\paragraph{HSVI}
HSVI (heuristic search value iteration). Uses a forward search heuristic to find relevant beliefs and keeps lower and upper bounds on the 
optimal value function. The search heuristic makes use of the bounds on the current value function to decide which nodes of the belief tree to 
expand. To expand a node, an action and observation have to be chosen. The action is chosen which has the greatest upper bound. This
guarantees to find the optimal action. If the action is suboptimal the upper bound will decrease, making another action preferable.
Evaluated on the \textbf{rock sample} problem: Only some of the rocks have scientific value. The status of the rock is known since
we don't know what kind of observation we will get. Sampling a rock is costly. Map is $n \times n$ discrete regions. Full state space
is cross product with the binary feature type {good,bad}. The algorithm has a smilar performance than (classical) PBVI and in one 
case (Tag problem did significantly better).

\paragraph{FSVI}
Forward search value iteration. Says HSVI is based but the keeping the upper bounds drastically reduces the speed of the algorithm. 
Suggests a new method for belief point selection and for the ordering of the backups. Use the solution of an MDP as an initial 
guess. Actions are selected both on the optimal policy for the MDP. Generate sample trajectories were simulations trials are executed.
Only the states in the trajectory are updated. Need good heuristic to explore the state space. An important drawback of using the underlying
MDP is the inability of the MDP to assess actions taht gather information. The heuristic of FSVI to traverse the belief space is based on 
the optimal solution to the underlying MDP. Sometimes FSVI needs another heuristic to allow for action gather actions to be tried.
Basically it is much faster than HSVI. It is however not able to deal with long information gathering sequences.

\paragraph{SARSOP}: Efficient Point-Based POMDP Planning by Approximating Optimal reachable belief space. Compute successive approximations
of the reachable belief set $\mathcal^*{R}(b_0)$. Considering an initial belief starting point. Improves sampling over time through 
simple on-line learning techniques. Tries to sample directly the optimal reachable space through learning-enhances exploration and bounding 
techniques. Keeps lower and upper bound on the value function. Set the initial belief point $b_0$ as the root and sample trees. SARSOP
uses the sawtooth approximation. The upper bound can be initialised in various ways [MDP or the Fast Informed Bound technique].
Use upper and lower bound to bias the sampling toward $R^*$. Shows significant speed improvement on some problems, but the achieved reward
is very similar. Can do state space in $10'000$, $8$ actions, $1000$ observations (Integrated Exploration).

\paragraph{Factored Value Function Approximation}
Our work focuses on optimising a basic operation -the backup of a single belief point-. Consider a factored POMDP model.
It is represented by a two-stage dynamic Bayesian network (DBN). Say that the value function of the MDP is linear. When 
the value function is linear, it is compact to do a bellman backup operation (Only for on policy).


\subsection{Continuous PBVI}

\paragraph{Ca-Perseus} 

\cite{Spaan05icra} % 
Consider continuous actions in a PBVI setting, the task is active localisation. Robot with omnidirectional vision.
The back up operation $HV_n$ requires linear programming and is there for costly in high dimensions [Planning and acting in a 
partially observable stochastic domain]. Exploit the sparsity of the belief simplex, then only a few points are relevant.
The key idea is that, in each value iteration step, we can improve the value of all points in the belief set by only updating
the value and its gradient. Action is parametrised by problem-specific parameters.  
Replace the maximisation in the bellman backup operator by sampling. Do an approximate bellman backup. We used a belief set of 
$10'000$.
\begin{enumerate}
 \item Robot randomly explores the environment collecting samples
 \item do random backups 
\end{enumerate}


\paragraph{Point-based Value Iteration for Continuous POMDPs}

\cite{PBVI_C_2006}

observation, transition and reward are modelled using Gaussian mixtures. The belief is 
represented by using Gaussian mixtures or particle sets. Show that the bellman backup
operator is represented in closed form. Extend Perseus to deal with continuous action 
and observations through using appropriate sampling techniques. Show that the PWLC
property of POMDP holds for the continuous state, discrete action,observation case.
Prove that value function for continuous-state pomdp is convex, and then PWLC.
Have $\alpha$-functions instead of $\alpha$-vectors. Experiment: simulated robotic domain.
space is continuous in 1D, for doors. 

\paragraph{MCVI}

\cite{MCVI_CS_POMDPs}

\paragraph{MC-PPOMDP}

\cite{mc_update_ppomdps}


\section{Belief space planning}






\paragraph{Sequential Trajectory Re-planning with Tactile IG for Dexterous Grasping under object-pose Uncertainty}

§1 Known object mode, §2 precomputed grasp and pregrasp configurations are known a priori for this object. §3 unreliable estimate of the object's pose.
Given belief over object pose (estimated via PF) a reach-to-grasp trajectory is planned. 
Uses the non-gaussian belief planning method, our algorithm plans trajectories assuming only the maximum likelihood observations given the current belief state.
Do what is known as PoseSLAM -> figure out 6DOF 


\paragraph{Task-driven Tactile Exploration}
  
World relative trajectories: arm trajectories parametrized with respect to a pose of an object in the world.
Execute WRT with object pos which is most likely to succeed. 
  
\cite{Hsiao_RSS_10} % Task-driven Tactile Exploration




%\begin{table}[!ht]
%  \centering
%  \begin{tabular}{l|l|c|c|c|c|c|c|c|}%
%	& name 				 	& S & A & O & B & online & offline	 	\\ \hline
%\multirow{3}{4em}{Multiple row}	PBVI\cite{PBVI}     		 	& d & d & d & d & \checkmark  &         	\\ \hline%
%	SARSOP\cite{SARSOP} 		 	      	\\ \hline
%		& c & c & c & c (pf) & 	       &  \checkmark	\\ \hline 
%			 	& c & d & d & c (pf) & \checkmark     &  \\ \hline 	
%  
%  \end{tabular}
%  \caption{Summary of all POMDP methods}
%\end{table}


\bibliographystyle{plainnat} 
\bibliography{citations/pomdp.bib}


\end{document}
